\usepackage{hyperref}
% Define footnote color option.
\makeatletter
\def\@footnotecolor{red}
\define@key{Hyp}{footnotecolor}{%
 \HyColor@HyperrefColor{#1}\@footnotecolor%
}
\def\@footnotemark{%
  \leavevmode
  \ifhmode\edef\@x@sf{\the\spacefactor}\nobreak\fi
  \stepcounter{Hfootnote}%
  \global\let\Hy@saved@currentHref\@currentHref
  \hyper@makecurrent{Hfootnote}%
  \global\let\Hy@footnote@currentHref\@currentHref
  \global\let\@currentHref\Hy@saved@currentHref
  \hyper@linkstart{footnote}{\Hy@footnote@currentHref}%
  \@makefnmark
  \hyper@linkend
  \ifhmode\spacefactor\@x@sf\fi
  \relax
}%
\makeatother
\hypersetup{colorlinks=true,footnotecolor=blue}

\usepackage{amsmath}

% Used for printing a trailing space better than using a tilde (~) using the
% \xspace command
\usepackage{xspace}
% Print text in maroon
\newcommand{\hlred}[1]{\textcolor{Maroon}{#1}}
% Command to print a very short space
\newcommand{\hairsp}{\hspace{1pt}}
% Command to print i.e.
\newcommand{\ie}{\textit{i.\hairsp{}e.}\xspace}
% Command to print e.g.
\newcommand{\eg}{\textit{e.\hairsp{}g.}\xspace}

% Set up the images/graphics package
\usepackage{graphicx}
\setkeys{Gin}{width=\linewidth,totalheight=\textheight,keepaspectratio}
\graphicspath{{graphics/}}

\usepackage{enumitem}
\def\labelitemi{--}

% The following package makes prettier tables.  We're all about the bling!
\usepackage{booktabs}

% The units package provides nice, non-stacked fractions and better spacing
% for units.
\usepackage{units}
\usepackage[english]{babel}
\usepackage[latin1]{inputenc}

% The fancyvrb package lets us customize the formatting of verbatim
% environments. We use a slightly smaller font.
\usepackage{fancyvrb}
\fvset{fontsize=\normalsize}

% Small sections of multiple columns
\usepackage{multicol}
